\documentclass{scrartcl}
\usepackage[ansinew]{inputenc}
%\usepackage[T1]{fontenc}
%\usepackage[ngerman]{babel} % Neue Rechtschreibung
\usepackage{amsmath} % Verbesserter Mathesatz
\usepackage{amssymb}
\usepackage{graphicx}

\parskip1ex
\title{\begin{normalsize}Manual Addendum to the ParticleTracker\_ Manual for the\end{normalsize} \\ ParticleTracker\_3D - ImageJ PlugIn}
\date{June 2008 (revised November 2010)}
\author{Janick Cardinale, \texttt{janickc@inf.ethz.ch}}
\begin{document}

\maketitle

\section{Introduction}
\label{sec:intro}
The ImageJ plugin \texttt{ParticleTracker\_} version 1.2 is a straight forward extension to three space dimensions of the ImageJ Plugin ParticleTracker\_ written by Guy Levy. \textbf{This version handles both, 2D movies and 3D movies. Also, the plugin might be used for feature point detection only in 2D and 3D.} Both versions can be downloaded from the Mosaic\footnote{http://www.mosaic.ethz.ch/Downloads/index} web-page. 

In this addendum, changes for the user to the user manual of the \texttt{ParticleTracker\_} plugin are described. Generally, the extension is designed such that it does not affect the handling of the plugin. At the moment, there is no visualization of the z-coordinate. Only z-projected trajectories can be displayed.

\section{Extensions and Changes}
\label{sec:pluginstart}
\subsection{Installation}
Please note that the plugin needs a Java Runtime Environment (JRE) version 5 or higher and ImageJ version 1.36b or higher. To install the plugin, copy the \texttt{mosaic\_plugins.jar} archive to ImageJ's plugin folder and restart ImageJ. After restarting ImageJ, the plugin should appear in the \texttt{plugins$\rightarrow$Mosaic} submenu.

\subsection{Load movie and start}
Open a 4D image stack. \textbf{In the image properties dialog (shortcut \texttt{Ctrl-p}) check if the number of slices and frames is set correctly}.

% If the plugin is started, a new selection for the background subtraction appears. There are 4 possible choices:

% \begin{itemize}
% 	\item \textbf{None}: The background is not removed. Only the noise reduction operation as described (for the 2D case) in \cite{sbalzarini} is applied. This option might be useful to better handle the absolute threshold parameter if long wavelength modulations were removed in a prior operation. 
% 	\item \textbf{Boxcar}: The background is removed slice by slice using the average intensity of a sliding window. See \cite{sbalzarini}. This is the fastest option.
% 		\item \textbf{Background Subtractor}: This option uses a histogram based background subtractor implemented in the \texttt{BackgroundSubtractor\_} Plugin. It is not that fast as the boxcar method but should better separate the background from the signal. It is also more restrictive in terms of subtracting signal. You may use this option if your images have a low signal to noise ratio (SNR). For further information, please see the appropriate tutorial.
% 	\item \textbf{Laplace Operator}: The background is removed using the discrete Laplace operator on a grid. If you use this mode, not the amount of intensities is the determining factor in the particle discrimination step, but the form (slope) of the feature point. 
% \end{itemize}

% Further, there is a new option to choose an absolute threshold instead of the percentile parameter. This might help if you have neglectable bleaching effects and/or if you have an inconstant amount of objects to track. Please note that an absolute threshold value might change after the preprocessing steps of the algorithm. A workaround to this effect is to manually subtract the background first and select \texttt{none} as a preprocessing parameter. Subtracting the background can be done either with the built in \texttt{BackgroundSubtractor\_} plugin that comes with this plugin.

\subsection{Visualization}
\label{sec:visualizing}
If you like to only detect particles (not link them), click on the \texttt{Save Detected} button. The detected particles are shown if you slide through the image stack using the sliding bar on the parameter window. 

Note that, while visualizing the trajectories, a frame consists of $n$ slices. If you would like to export movies, select a trajectory or an area and focus on your selection using the appropriate button in the result window. This generates  a RGB image that you may z-project using the Image5D framework for ImageJ.

\subsection{Batchprocessing}
The plugin supports batch processing using standard ImageJ macros. If the plugin detects that a macro is running, data are written to disk (if the user has the appropriate permissions) instead of visualized in a new window. A directory \texttt{ParticleTracker3DResults/} is created in the directory of the image. The result file has the same name as the file name of the image, concatenated with the suffix \texttt{PT3D.txt}. If there are any problems while writing the data to the disk, the macro is aborted.

\subsection{Results}
\label{sec:results}
\paragraph{Trajectory information format}
This format is saved to disk if you saved a report or if the report was printed to the text panel. The output from the 2D version differs only slightly from the the output of newest 3D version differs: the z-coordinate, the first, third, and fourth intensity momentum was added. The z-coordinate starts from the first slice with value 1 and ends with the number of slices per frame of your movie. If the image to process is in 2D, the z-coordinate is set to 0.0. The resulting text format is then:
\begin{table}[ht]
		%\centering
		\ttfamily
		\begin{tabular}{*{10}{l}}	
			\multicolumn{10}{l}{ \%Trajectory n} \\		
			f	&	x & y & z & I$_0$ & I$_1$ & I$_2$ & I$_3$ & I$_4$ & s					
		\end{tabular}
\end{table}

with
\begin{table}[ht]
	\centering
	\begin{tabular}{cp{10cm}}
		\texttt{x} & The x coordinate of the particle in pixel from left to right \\
		\texttt{y} & The y coordinate of the particle in pixel from top to down \\
		\texttt{z} & The z coordinate of the particle in slices from the first to the last slice of the frame\\
		\texttt{I}$_k$ & The intensity momentum of order k \\
		\texttt{s} & The particles score in the non particle discrimination \\
	\end{tabular}
\end{table}

\paragraph{Particles information (Frame) format}
This format is written to disk if one detected particles only (by clicking the button ``Save detected'' in the parameter dialog). It is also the format that can be read by the plugin. The format does not contain any information about trajectories. Using the same shortcuts as in the paragraph above, the frame format is:
\begin{flushleft}
\begin{table}[h]		
		\ttfamily
		\begin{tabular}{*{8}{l}}	
			\multicolumn{5}{l}{frame f} \\		
			x & y & z & I$_0$ & I$_1$ & I$_2$ & I$_3$ & I$_4$ 
		\end{tabular}
\end{table}
\end{flushleft}
The numbers are separated by the white-space character.
\paragraph{Full report format}
The header is supplemented by the line (the right arrow $\rightarrow$ stands for a tab character): 
\begin{flushleft}
\ttfamily
\begin{tabbing}
\%frame\= \kill \\
\%Frames information \\
\%$\rightarrow$\>Depth: s slices
\end{tabbing}
\end{flushleft}
where \texttt{s} represents the amount of slices per frame. 
%If \texttt{Absolute Threshold} is used, the percentile parameter is replaced by the line:
% \begin{flushleft}
% \ttfamily
% \begin{tabbing}
% \%frame\= \kill \\
% \%Frames information \\
% \%$\rightarrow$\>Absolute threshold: x
%\end{tabbing}
%\end{flushleft} 
The following information are in the same format as in the 2D version except that for the particles the z coordinate is appended:
\begin{flushleft}
\ttfamily
\begin{tabbing}
\%\=12345\=12345\=12345\=y12345\=z \kill
\%\>Frame f:\\
\%\>$\rightarrow$\>n particles found\\
\%\>$\rightarrow$\>Detected particle positions: \\
\%\>$\rightarrow$\>$\rightarrow$\>x$\rightarrow$\>y$\rightarrow$\>z
\end{tabbing}
\end{flushleft}
\paragraph{Export to ImageJ Table}
Particles and trajectory information can be exported to a ImageJ-compatible format using ImageJ tables. In the result window, you can export particle or trajectory information using the appropriate button:
\begin{itemize}
\item Table content when exporting particles to ImageJ tables:\\
  \begin{table}[ht]
    % \centering
    \ttfamily
    \begin{tabular}{*{11}{l}}	
      row\_nb & frame & x & y & z & I$_0$ & I$_1$ & I$_2$ & I$_3$ & I$_4$ & s					
    \end{tabular}
  \end{table}
\item Table content when exporting trajectories to a ImageJ table:\\
  \begin{table}[ht]
    % \centering
    \ttfamily
    \begin{tabular}{*{12}{l}}	
      row\_nb & trajectory\_nb & frame & x & y & z & I$_0$ & I$_1$ & I$_2$ & I$_3$ & I$_4$ & s					
    \end{tabular}
  \end{table}
\end{itemize}
\section{Acknowledgments}
Many thanks to Dr. Kota Miura, Centre for Molecular and Cellular Imaging, EMBL, for contributing the export to ImageJ table feature.


\bibliographystyle{abbrv}
\bibliography{refs}

\end{document}

